\section{Аналитическая часть}

\subsection{Объекты сцены}\label{scene}

Сцена состоит из нескольких обхектов.

\textbf{Источник света} -- объект сцены; точка в пространстве, от которой свет излучается равномерно во все направления.

Формула вектора в трехмерном пространстве~\cite[с.~5--7]{beklemishev1984geometry}:
\begin{equation}
\vec{v} = \begin{pmatrix} x \\ y \\ z \end{pmatrix}.
\end{equation}

\textbf{Плоскость} -- непрозрачный объект сцены; горизонтальная плоскость, которая служит основой для расположения объектов в сцене.

Уравнение плоскости в трехмерном пространстве можно записать в общем виде следующим образом~\cite[с.~45]{beklemishev1984geometry}.
\begin{equation}
	Ax + By + Cz + D = 0,
\end{equation}\label{plane}
где: 

\begin{enumerate}[label=---]
	
	\item $A, B, C$ -- коэффициенты, определяющие ориентацию плоскости; соответствуют нормальному вектору $\vec{n}$ к плоскости;
	
	\item $D$ -- свободный член, который смещает плоскость относительно начала координат;
	
	\item $x, y, z$ -- координаты произвольной точки на плоскости.
	
\end{enumerate}


\textbf{Сосуд} и \textbf{жидкость} -- прозрачные объекты сцены, представленные как прямые цилиндры, причем сосуд -- тонкостенный.

Уравнение прямого цилиндра~\cite[с.~45]{beklemishev1984geometry} с радиусом $r$, высотой $h$ и коэффициентом масштабирования $m$:

\begin{equation}
	x^2 + m^2y^2 = r^2, 0 \le z \le h.
\end{equation}\label{cyl}

\textbf{Стержень} -- непрозрачный объект сцены, представленный как прямой параллелепипед.

Параллелепипед можно задать шестью уравнениями плоскости~\cite[с.~143]{lengyel2011mathematics}:

\begin{align}
	x &= 0, \quad x = r_x, \\
	y &= 0, \quad y = r_y, \\
	z &= 0, \quad z = r_z,
\end{align}\label{box}
где $r_x$, $r_y$ и $r_z$ представляют его размеры.

Cтержень и жидкость располагаются внутри сосуда, причем стержень частично погружен в жидкость, располагаясь внутри его объема от верхней поверхности до определенной глубины.

\subsection{Способы представления объектов сцены}

Существует несколько способов представления объектов, которые различаются по степени сложности, возможностям редактирования и применяемости в разных сферах трехмерной графики. Рассмотрим три основных типа представления объектов, описанные в~\cite[c.~102]{vaughan2012digital}

\subsubsection{Каркасные модели}

Каркасная модель -- способ представления объекта, при котором отображаются только его ребра и вершины, без внутренних поверхностей. Является основой для дальнейшего моделирования, текстурирования и визуализации. Каркасная модель не включает в себя полные геометрические данные (поверхности), а только контуры, которые дают общее представление о форме объекта~\cite[С.~102, 107]{vaughan2012digital}.

\subsubsection{Полигональные модели}

Полигональная модель -- тип модели, который состоит из набора точек (вершин), ребер и полигонов, которые объединяются в единую структуру. Набор поверхностей можно задать аналитически (уравнением или системой уравнений) либо в виде полигональной сетки, где каждый полигон представляет собой многоугольник, который обычно состоит из треугольников или четырехугольников~\cite[c. 108]{vaughan2012digital}.

\subsubsection{Твердотельные модели}

Твердотельная модель --  это метод представление объекта, который включает полное описание его геометрической формы и внутренней структуры. В отличие от поверхностных моделей, которые описывают только внешнюю оболочку объекта, твердотельная модель полностью описывает объём и физические характеристики объекта~\cite{shapiro2001solid}.

\subsubsection*{Выбор представления объектов сцены}

Так как каркасные и полигональные модели не могут учитывать физические свойcтва объектов, выбрана твердотельная модель, как способ представления объектов сцены, который обеспечивает полное описание объекта, что позволяет учитывать физические свойства, такие как отражение и преломление света, которые необходимы для построения реалистичного изображения.

\subsection{Методы рендеринга сцены}

Рендеринг -- это процесс преобразования трехмерной модели или сцены в изображение, которое может быть отображено на экране~\cite[C.~11--12]{moller2018real}.

При выборе метода рендеринга моделей, необходимо учитывать поддержку эффектов освещения, отражений и преломлений выбранным методом, следовательно, будут рассмотрены методы трассировки лучей.

Трассировка лучей -- это метод рендеринга, который моделирует путь света от камеры или источников света в сцене для расчета цветов пикселей. Рассмотрим следующие методы трассировки лучей, описанные в~\cite[C.~443--446]{moller2018real}.

\subsubsection{Метод <<бросания лучей>>}

Метод <<бросания лучей>> -- это базовый метод трассировки лучей, при котором для каждого пикселя на экране из камеры проходит луч в сцену, и определяется, с каким объектом он пересекается. Луч проверяет, есть ли на его пути объекты, и возвращает информацию о ближайшем объекте. Этот метод используется в основном для вычисления видимости и освещенности.

Данный метод обладает следующими характеристиками.

\begin{itemize}
	\item Трудоемкость: $O(N)$, где $N$ -- количество пикселей.
	\item Требуется память для хранения информации о пересечении луча с объектами (объекты сцены, буфер изображения).
	\item Ограничение основным освещением, то есть без учета отражений, преломлений лучей и сложных эффектов.
\end{itemize}

\subsubsection{Обратная трассировка лучей}

Обратная трассировка лучей -- это улучшенная версия метода <<бросания лучей>>, которая учитывает не только пересечение луча с объектами, но и дополнительные физические явления, такие как отражения, преломления и тени. Каждый луч может быть использован для вычисления того, как свет распространяется в сцене, создавая более реалистичное изображение.

Данный метод обладает следующими характеристиками.

\begin{itemize}
	\item Трудоемкость: $O(N \cdot R)$, где $N$ -- количество пикселей, $R$ -- количество отражений и преломлений лучей.
	\item Требуется память для хранения информации о пересечении луча с объектами (объекты сцены, буфер изображения), а также дополнительная память для хранения информации о преломлении и отражении каждого луча.
	\item Учитывает основное освещение, отражения и преломления лучей.
\end{itemize}

\subsubsection{Трассировка путей}

Трассировка путей -- это улучшение трассировки лучей, где для каждого луча строится случайный путь через сцену. Лучи могут многократно отражаться и преломляться, что позволяет точно моделировать более сложные эффекты, такие как мягкие тени и рассеяние света.

Данный метод обладает следующими характеристиками.

\begin{itemize}
	\item Трудоемкость: $O(N  \cdot M)$, где $N$ -- количество пикселей, $M$ -- количество дополнительных случайных путей на пиксель.
	\item Требуется память для хранения информации о пересечении луча с объектами (объекты сцены, буфер изображения), а также дополнительная память хранения путей, случайных точек пересечений и анализа их взаимодействий.
	\item Поддерживается глобальное освещение, мягкие тени, преломления, отражения и рассеяние.
\end{itemize}

\subsubsection*{Выбор метода рендеринга моделей}

В таблице~\ref{table:rt} приведено сравнение методов трассировки лучей по вычислительной нагрузке, реалистичности и сложности реализации.

В отличие от метода <<бросания лучей>>, который только определяет пересечения лучей с объектами, обратная трассировка лучей способна точно моделировать такие эффекты, как отражения, преломления и тени. 

Трассировка путей предоставляет максимальную реалистичность, но требует значительно большего времени на обработку из--за необходимости генерировать множество путей для каждого луча. Этот метод может быть слишком ресурсоемким для большинства приложений, где требуется высокая скорость рендеринга, но при этом важна качественная симуляция отражений и преломлений.

\begin{table}[h!]
	\centering
	\small
	\renewcommand{\arraystretch}{1.5}
	\caption{Сравнение методов трассировки лучей}
	\begin{tabular}{|>{\raggedright\arraybackslash}m{3.8cm}|>{\raggedright\arraybackslash}m{3.8cm}|>{\raggedright\arraybackslash}m{3.8cm}|>{\raggedright\arraybackslash}m{3cm}|}
		\hline
		\textbf{Метод} & \textbf{Вычислительная нагрузка} & \textbf{Реалистичность изображения} & \textbf{Сложность реализации} \\ \hline
		<<Бросание лучей>> & Низкая & Низкая & Низкая \\ \hline
		Обратная трассировка лучей & Средняя & Средняя & Средняя \\ \hline
		Трассировка путей & Высокая & Высокая & Высокая \\ \hline
	\end{tabular}
	\label{table:rt}
\end{table}

На основе анализа методов рендеринга моделей выбран метод обратной трассировки лучей для построения реалистичного изображения с учетом преломлений и отражений, так как предлагает более сбалансированное решение между качеством изображения и вычислительными затратами.

\subsection{Модели освещения}

Модели освещения -- это математические подходы, используемые в компьютерной графике для расчета взаимодействия света с поверхностями объектов. Они играют ключевую роль в создании реалистичных изображений, моделируя такие эффекты, как отражение, рассеивание и тени. В зависимости от сложности, каждая модель имеет свои характеристики, применимость и вычислительные затраты. Рассмотрим основные модели освещения, описанные в~\cite{Zadorozhny2020}.

\subsubsection{Простая модель освещений}

Простая модель освещения основывается на расчете диффузного освещения, где свет поступает равномерно на поверхность. Это базовая модель, использующая информацию о нормалях поверхностей и источниках света, но не учитывающая эффекты отражений, преломлений или блеска. Вычисления для данной модели быстрые и несложные, но она имеет низкую реалистичность, так как не моделирует более сложные эффекты освещения~\cite[c.~35]{Zadorozhny2020}.

\subsubsection{Модель освещения Гуро}

Модель Гуро улучшает простую модель освещения за счет учета нормалей в вершинах объектов. Она использует интерполяцию значений освещенности между вершинами полигона для получения более плавного перехода освещенности на поверхности. Модель Гуро хорошо подходит для рендеринга объектов с матовыми или полугладкими поверхностями, но она ограничена в точности и не может точно моделировать отражения и блики~\cite[c.~43]{Zadorozhny2020}.

\subsubsection{Модель освещения Фонга}

Модель Фонга является более сложной, но и наиболее точной в плане реалистичности. Она учитывает три компонента освещения: диффузный, зеркальный и окружающий свет, что позволяет получать точные блики и эффекты отражения. Модель Фонга позволяет моделировать как гладкие, так и зеркальные поверхности. Вычисления для каждого пикселя значительно сложнее, так как включают расчет угла между наблюдателем и источниками света, а также интенсивности отражений. Это требует более высоких вычислительных мощностей, но позволяет получать очень реалистичные результаты~\cite[c.~33]{Zadorozhny2020}.

\subsubsection*{Выбор модели освещения}

В таблице~\ref{table:lt} приведено сравнение моделей освещения по вычислительной нагрузке, реалистичности и сложности реализации.

\begin{table}[h!]
	\centering
	\small
	\renewcommand{\arraystretch}{1.5}
	\caption{Сравнение моделей освещения}
	\begin{tabular}{|>{\raggedright\arraybackslash}m{3.8cm}|>{\raggedright\arraybackslash}m{3.8cm}|>{\raggedright\arraybackslash}m{3.8cm}|>{\raggedright\arraybackslash}m{3cm}|}
		\hline
		\textbf{Модель освещения} & \textbf{Вычислительная нагрузка} & \textbf{Реалистичность изображения} & \textbf{Сложность реализации} \\ \hline
		Простая & Низкая & Низкая & Низкая \\ \hline
		Гуро & Средняя & Средняя & Средняя \\ \hline
		Фонга & Высокая & Высокая & Средняя \\ \hline
	\end{tabular}
	\label{table:lt}
\end{table}

 На основе анализа моделей освещения выбрана модель освещения Фонга. Простая модель недостаточна из-за низкой реалистичности, а модель Гуро не обеспечивает высокого уровня детализации, а также не подходит к твердотельной модели, геометрическая форма которой задается аналитически. Модель Фонга, благодаря расчету отраженного света с учетом бликов, обеспечивает более реалистичное изображение при умеренной вычислительной нагрузке.

\subsection*{Вывод}

В данном разделе были рассмотрены подходы к созданию реалистичных изображений. Описаны характеристики объектов сцены и методы их представления, проанализированы различные алгоритмы рендеринга и модели освещения. На основании анализа сделан выбор в пользу твердотельной модели представления объектов, модели освещения Фонга и алгоритма прямой трассировки лучей. Эти методы наиболее подходящие для разработки алгоритма, способного реалистично отображать сцены с учетом отражений, преломлений и сложного взаимодействия света.