\phantomsection\section*{ЗАКЛЮЧЕНИЕ}\addcontentsline{toc}{section}{ЗАКЛЮЧЕНИЕ}

В ходе выполнения курсовой работы было разработано программное обеспечение для создания реалистичных изображений стержня, частично погруженного в прозрачный сосуд, наполненный жидкостью, с применением алгоритма обратной трассировки лучей. Описание моделируемых объектов, выбор методов представления объектов и моделей освещения позволили обеспечить точное и реалистичное отображение сцены. Было обосновано использование алгоритма обратной трассировки лучей как основного инструмента для вычисления взаимодействия света с материалами, что позволило добиться высокого уровня детализации и достоверности изображения.

Реализация программы позволила провести ряд исследований для оценки производительности в зависимости от глубины рекурсии и использования дополнительных потоков. Результаты показали, что увеличение глубины рекурсии улучшает реалистичность изображения, но сопровождается увеличением времени построения кадра. Использование многозадачности с дополнительными потоками позволяет существенно сократить время рендеринга, что подтверждает важность эффективного распределения вычислительных ресурсов.

Кроме того, было проведено исследование влияния коэффициента преломления на визуализацию прозрачных объектов, что продемонстрировало, как изменение оптических свойств материалов влияет на визуальное восприятие сцены.

В результате работы выполнены все поставленные задачи, и достигнута цель данной курсовой работы.