\phantomsection\section*{ВВЕДЕНИЕ}\addcontentsline{toc}{section}{ВВЕДЕНИЕ}

В настоящее время компьютерная графика является одним из наиболее стремительно расширяющихся секторов в сфере информационных технологий. Ее область применения чрезвычайно широка, начиная от инженерии и науки, заканчивая искусством и медициной.

Одним из главных направлений компьютерной графики является создание реалистичных трехмерных изображений, которые могут быть использованы для наглядного представления разнообразной информации~\cite{peddie2013history}.

Цель курсовой работы -- разработка программного обеспечения для построения реалистичного изображения стержня, частично погруженного в тонкостенный прозрачный сосуд, наполненный жидкостью.

Для достижения цели курсовой работы необходимо выполнить следующие задачи: 

\begin{enumerate}[label=\arabic*)]
	
	\item описать моделируемые объекты сцены;
	
	\item проанализировать способы представления объектов и обосновать выбор наиболее подходящего;
	
	\item проанализировать модели освещения и обосновать выбор наиболее подходящей;
	
	\item проанализировать алгоритмы рендеринга сцены и обосновать выбор наиболее подходящего;
	
	\item разработать выбранные алгоритмы;
	
	\item реализовать программное обеспечения для достижения цели данной работы;
	
	\item исследовать время построения изображения в зависимости от глубины рекурсии и от количества используемых дополнительных потоков; реалистичность преломления света для прозрачных объектов в зависимости от входные параметров для материалов объектов для разработанного программного обеспечения.
	
\end{enumerate}


